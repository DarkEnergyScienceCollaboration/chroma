%%%%%%%%%%%%%%%%%%
% DOCUMENT CLASS %
%%%%%%%%%%%%%%%%%%
\documentclass[apj]{emulateapj}

\usepackage{times}
\usepackage{amsmath}
\usepackage{commath}
\usepackage[utf8x]{inputenc}
\urlstyle{same}
\usepackage{hyperref}
\hypersetup{
  pdftitle={Shortcomings of the single Sersic galaxy profile for
    estimating chromatic effects with a ring test},
  pdfauthor={Joshua Meyers},
  colorlinks=true,
  linkcolor=blue,
  citecolor=blue,
  urlcolor=blue,
  bookmarksnumbered
}

\pagestyle{myheadings}

\shorttitle{Single Sersic ring test}
\shortauthors{Meyers et al.}

\begin{document}

\title{Shortcomings of the single Sersic galaxy profile when estimating chromatic weak lensing biases with a ring test}

\author{
J.~Meyers\altaffilmark{1}}
\email{jmeyers3@stanford.edu}

\altaffiltext{1}{Department of Physics, Stanford University, Stanford, CA 94305}

%Abstract
\begin{abstract}
This is the abstract.  Lorem ipsum dolor sit amet, consectetuer adipiscing elit. Donec hendrerit tempor tellus. Donec pretium posuere tellus. Proin quam nisl, tincidunt et, mattis eget, convallis nec, purus. Cum sociis natoque penatibus et magnis dis parturient montes, nascetur ridiculus mus. Nulla posuere. Donec vitae dolor. Nullam tristique diam non turpis. Cras placerat accumsan nulla. Nullam rutrum. Nam vestibulum accumsan nisl.
\end{abstract}

%intro
\section{Introduction}\label{sec:intro}
Cosmic shear experiments aim to constrain cosmological parameters by
measuring the small departure from statistical isotropy of distant
galaxy shapes induced by the gravitational lensing from foreground
large scale structure.  The shapes of galaxy images collected from
telescopes, however, are not only affected by cosmic shear (typically
a $\sim$ 1\% effect), but also by the combined point spread function
(PSF) of the atmosphere (for ground-based experiments), telescope
optics, and the image sensor (often a few \% effect).  The shape of
this additional convolution kernel is typically constrained from the
shapes of stars, which are effectively point sources before being
smeared by the PSF.  Galaxy images can then be deconvolved with the
estimated convolution kernel.  Implicit in this approach is the
assumption that the galactic kernel is the same as the stellar kernel.
Effects that make the PSF dependent on wavelength will violate this
assumption, as stars and galaxies at different redshifts, have
different spectral energy distributions, and hence different PSFs.

Examples of wavelength-dependent PSF contributions include:

\begin{itemize}
  \item atmospheric differential chromatic refraction (DCR)
  \item atmospheric seeing
  \item telescope optics DCR
  \item photoconversion depth in the sensor and subsequent charge diffusion
\end{itemize}

This note will focus on PSF mis-estimations due to the first of the
above effects, though the conclusions are applicable to any situation
in which the bias due to using the wrong PSF is estimated using a ring
test with a simple model parameterization.

\section{Analytic Expectations}\label{sec:analytic}
\citet{Plazas2012} derived analytic expressions for the bias expected
in weak lensing measurements due to differential chromatic refraction.
I will summarize the main points of their argument here.

Let $R(\lambda, z_a)$ be the refraction towards the zenith of a photon
with wavelength $\lambda$ and true zenith angle (before refraction)
$z_A$.  The refraction can be factored into
\begin{equation}
  R(\lambda; z_a) =  g(\lambda) \mathrm{tan}(z_a)
  \label{eqn:Rlamza}
\end{equation}
and $g(\lambda)$, which implicitly depends on air pressure,
temperature, and the partial pressure of water vapor, and can be
obtained from \citet{Edlen1953} and \citet{Coleman1960}.  On the focal
plane, and forr monochromatic sources, the only effect is to move the
apparent position of the object.  For sources which are not
monochromatic (i.e. all real sources), having a wavelength density of
surviving photons (i.e. the product of the source photon density and
the total system throughput function) given by
$\mathrm{p}_{\lambda}(\lambda)$, the mismatched displacements of
different wavelengths introduces a convolution kernel in the zenith
direction that can be written in terms of the inverse of Equation
\ref{eqn:Rlamza}:
\begin{equation}
  h(R) = \frac{p_\lambda(\lambda(R; z_a)) \left|\od{\lambda}{R}\right|}{\int{p_\lambda(\lambda) \dif{\lambda}}}
  \label{eqn:convker}
\end{equation}
This kernel can largely be characterized in terms of its first and
second central moments given by:
\begin{equation}
  \bar{R} = \int{h(R) R \dif{R} } = \frac{\int{R(\lambda; z_a) p_\lambda(\lambda)\dif{\lambda} }}{\int{p_\lambda(\lambda) \dif{\lambda}}}
  \label{eqn:rbar}
\end{equation}
\begin{equation}
  V = \int{h(R) (R - \bar{R})^2 \dif{R} } = \frac{\int{(R(\lambda; z_a) - \bar{R})^2 p_\lambda(\lambda)\dif{\lambda} }}{\int{p_\lambda(\lambda) \dif{\lambda}}}
  \label{eqn:V}
\end{equation}

Galaxy shapes can be characterized by the quadrupole moments of their
light distribution given by:
\begin{equation}
  I_{\mu \nu} = \frac{1}{f}\int{\dif{x} \dif{y} I(x,y)\mu - \bar{\mu}(\nu - \bar{\nu})}
\end{equation}
\begin{equation}
  \bar{\mu} = \frac{1}{f}\int{\dif{x} \dif{y} I(x,y)\mu}
\end{equation}
\begin{equation}
  f = \int{\dif{x} \dif{y} I(x,y)}
\end{equation}

In particular, galaxy size and 2-component ellipticity are frequently defined as:
\begin{equation}
  r^2 = I_{xx} + I_{yy}
\end{equation}
\begin{equation}
  e_1 = \frac{I_{xx} - I_{yy}}{r^2}
\end{equation}
\begin{equation}
  e_2 = \frac{2 I_{xy}}{r^2}
\end{equation}

For the case of differential chromatic refraction, we can set, without
loss of generality, the $x$ direction to be towards zenith.  The
effect of DCR is then to take $I_{xx} \rightarrow I_{xx} + V$, which
also takes $r^2 \rightarrow r^2 + V$, but leaves $I_{yy}$ and $I_{xy}$
unchanged.  The ellipticity and size parameterizations are defined in
terms of the quadrupole moments before the galaxy light distribution
is smeared by the PSF, but we only have access to the light
distribution after convolution.  The second moments before ($I^g$) and
after ($I^o$) convolution are related via the second moments of the
PSF ($I^*$) like $I^g = I^o - I^*$.  If the SED's of galaxies and
stars were the same, then the effect of DCR would be to add $V$ to
both $I^o$ and $I^*$, which would then cancel when computing $I^g$ and
subsequently galaxy shape parameters.  The differences between stellar
and galactic SEDs, however, will introduce a small error $\Delta V \ll
r^2$ into $I_{xx}$ and $r^2$, which leads to biases in the derived
ellipticity parameters:
\begin{equation}
  e_1 \rightarrow \frac{I_{xx} + I_{yy} + \Delta V}{r^2 + \Delta V} \approx e_1 \left(1 + \frac{\Delta V}{r^2}\right) + \frac{\Delta V}{r^2}
\end{equation}
\begin{equation}
  e_2 \rightarrow \frac{2 I_{xy}}{r^2 + \Delta V} \approx e_2 \left(1 + \frac{\Delta V}{r^2}\right)
\end{equation}

Under the expectation that $\left<e_i\right> \approx 2\gamma_i$ and
defining the shear calibration parameters $m$ and $c$ such that
$\hat{\gamma} = \gamma (1 + m) + c$, where $\hat{\gamma}$ indicates
the estimator for the true shear $\gamma$, we reach the expectation
that $m$ = $\frac{\Delta V}{r^2}$, $c_1 = m/2$, and $c_2 = 0$.  Note
that nowhere in the above analysis is there any assumption on the
profile of the galaxy in question.

\section{Ring test}\label{sec:ringtest}
An alternative way to estimate the bias in $\hat{\gamma}$ induced by
DCR is to simulate galaxy images using the ``true'' (galactic) PSF and
then attempt to recover the simulated galactic ellipticities while
pretending that the PSF is ``wrong'' (stellar).  A ring test
\citep{Nakajima2007} is a specific prescription for such a suite of
simulations designed to rapidly converge to the correct value of
$\hat{\gamma}$.  The test gets its name from the arrangement of galaxy
shapes used in the simulated images, which form a ring in ellipticity
space (i.e. constant $|e|$), before any shear is applied.  By choosing
intrinsic ellipticities that exactly average to zero, the results of
the test converge faster than for randomly (but isotropically) chosen
ellipticities that only average to zero statistically.

The general procedure can be implemented as follows:

\begin{enumerate}
  \item Choose an input ``true'' reduced shear $g$
  \item Choose a pre-sheared ellipticity $e^s = (e_1^s, e_2^s)$
  \item Compute the sheared ellipticity from $e^o = \frac{e^s+g}{1+g^*e^s}$
  \item Generate a ``truth'' image by convolving the galaxy model with
    the ``true'' PSF (galactic in this case)
  \item Using a ``reconstruction'' PSF (stellar in this case), find
    the best fitting model to the ``truth'' image, record the measured
    ellipticity from the model parameters
  \item Repeat steps 3-5 using the opposite pre-sheared ellipticity
    $-e^s$
  \item Repeat steps 2-6 for as many values around the ellipticity
    ring as desired
  \item Average all recorded output ellipticity values.  This is the
    shear estimate $\hat{g}$
  \item Repeat steps 1-8 to map out the relation $g(\hat{g})$
  \item $m$ and $c$ are the slope and intercept of the best-fit linear
    relation between $g$ and $\hat{g}$ (note we assume that $g \approx
    \gamma$)
\end{enumerate}

As described above, the ring test requires some prescription for
creating galaxy images with a given ellipticity.  Here I investigate
using a single Sersic profile as the galaxy model.  The Sersic profile
has 7 parameters: the $x$ and $y$ coordinates of the center, the total
flux, the effective radius $r_e$ (also called the half-light radius),
the two component ellipticity $\mathbf{e}$, and the Sersic index $n$.
Using $r$ as an elliptical radial coordinate, the profile shape is:
\begin{equation}
  I(r) \propto \mathrm{e}^{[-k (r/r_e)^2]^{\frac{1}{2 n}}}
\end{equation}
The constant $k \approx 1.9992 n - 0.3271$ is chosen such that $r_e$
is the half-light radius for a circularized profile.  Limiting cases
of the Sersic profile include the Gaussian profile which has $n=0.5$,
an exponential profile which has $n=1.0$, and a de Vaucouleurs profile
which has $n=4.0$.  The profile ranges from smoothly peaked with small
tails at low $n$ (such as an $n=0.5$ Gaussian), to very sharply peak
with heavy tails (such as an $n=4.0$ deVaucoulers profile).

In addition to a galaxy model, the ring test also requires a model for
the PSF.  We have described above the PSF kernel that describes the
contribution of DCR, but other contributions to the PSF also exist,
due to atmospheric turbulence, telescope optics, and the detector for
example.  Ground based telescope PSFs are usually dominated by
atmospheric turbulence, and are frequently modeled by a Moffat
profile:
\begin{equation}
  I_p(r) \propto \left(1+\left(\frac{r}{\alpha}\right)^2\right)^{-\beta}
\end{equation}
\begin{equation}
  \alpha = \frac{\mathrm{FWHM}}{2\sqrt{2^{1/\beta}-1}}
\end{equation}
The PSF used in the ring test in this note is based on a Moffat
profile with a FWHM of $0$\farcs$7$ and softening parameter $\beta =
2.6$.  This base PSF is then convolved (in the zenith direction only)
with the DCR kernel given in Equation \ref{eqn:convker}.  The ``true''
PSF uses a DCR kernel derived from a galaxy SED, while the
``reconstruction'' PSF uses a DCR kernel derived from a stellar SED.

\section{Sersic index dependence}\label{sec:sersic}

\begin{figure}
\begin{center}
\epsscale{1.1}
\plotone{figures/all_pb12_ring_n.pdf}
\end{center}
\caption[fig2]{\label{fig:two} Lorem ipsum dolor sit amet,
  consectetuer adipiscing elit. Praesent libero orci, auctor sed,
  faucibus vestibulum, gravida vitae, arcu. Nunc posuere. Suspendisse
  potenti. Praesent in arcu ac nisl ultricies ultricies. Fusce
  eros. Sed pulvinar vehicula ante. Maecenas urna dolor, egestas vel,
  tristique et, porta eu, leo. Curabitur vitae sem eget arcu laoreet
  vulputate. Cras orci neque, faucibus et, rhoncus ac, venenatis ac,
  magna. Aenean eu lacus. Aliquam luctus facilisis augue. Nullam
  fringilla consectetuer sapien. Aenean neque augue, bibendum a,
  feugiat id, lobortis vel, nunc. Suspendisse in nibh quis erat
  condimentum pretium. Vestibulum tempor odio et leo. Sed sodales
  vestibulum justo. Cras convallis pellentesque augue. In eu magna. In
  pede turpis, feugiat pulvinar, sodales eget, bibendum consectetuer,
  magna. Pellentesque vitae augue.}
\end{figure}

Compare analytic expression with ring test results for different Sersic indices.

\section{Failure of ring test}\label{sec:failure}
Attempt to explain why the ring test fails for high Sersic index galaxies.

\begin{figure*}
\begin{center}
\epsscale{1.0}
\plotone{figures/case_study_n05.pdf}
\end{center}
\caption[fig1]{\label{fig:one} Lorem ipsum dolor sit amet,
  consectetuer adipiscing elit. Praesent libero orci, auctor sed,
  faucibus vestibulum, gravida vitae, arcu. Nunc posuere. Suspendisse
  potenti. Praesent in arcu ac nisl ultricies ultricies. Fusce
  eros. Sed pulvinar vehicula ante. Maecenas urna dolor, egestas vel,
  tristique et, porta eu, leo. Curabitur vitae sem eget arcu laoreet
  vulputate. Cras orci neque, faucibus et, rhoncus ac, venenatis ac,
  magna. Aenean eu lacus. Aliquam luctus facilisis augue. Nullam
  fringilla consectetuer sapien. Aenean neque augue, bibendum a,
  feugiat id, lobortis vel, nunc. Suspendisse in nibh quis erat
  condimentum pretium. Vestibulum tempor odio et leo. Sed sodales
  vestibulum justo. Cras convallis pellentesque augue. In eu magna. In
  pede turpis, feugiat pulvinar, sodales eget, bibendum consectetuer,
  magna. Pellentesque vitae augue.}
\end{figure*}

\begin{figure*}
\begin{center}
\epsscale{1.0}
\plotone{figures/case_study_n10.pdf}
\end{center}
\caption[fig1]{\label{fig:one} Lorem ipsum dolor sit amet,
  consectetuer adipiscing elit. Praesent libero orci, auctor sed,
  faucibus vestibulum, gravida vitae, arcu. Nunc posuere. Suspendisse
  potenti. Praesent in arcu ac nisl ultricies ultricies. Fusce
  eros. Sed pulvinar vehicula ante. Maecenas urna dolor, egestas vel,
  tristique et, porta eu, leo. Curabitur vitae sem eget arcu laoreet
  vulputate. Cras orci neque, faucibus et, rhoncus ac, venenatis ac,
  magna. Aenean eu lacus. Aliquam luctus facilisis augue. Nullam
  fringilla consectetuer sapien. Aenean neque augue, bibendum a,
  feugiat id, lobortis vel, nunc. Suspendisse in nibh quis erat
  condimentum pretium. Vestibulum tempor odio et leo. Sed sodales
  vestibulum justo. Cras convallis pellentesque augue. In eu magna. In
  pede turpis, feugiat pulvinar, sodales eget, bibendum consectetuer,
  magna. Pellentesque vitae augue.}
\end{figure*}

\begin{figure*}
\begin{center}
\epsscale{1.0}
\plotone{figures/case_study_n40.pdf}
\end{center}
\caption[fig1]{\label{fig:one} Lorem ipsum dolor sit amet,
  consectetuer adipiscing elit. Praesent libero orci, auctor sed,
  faucibus vestibulum, gravida vitae, arcu. Nunc posuere. Suspendisse
  potenti. Praesent in arcu ac nisl ultricies ultricies. Fusce
  eros. Sed pulvinar vehicula ante. Maecenas urna dolor, egestas vel,
  tristique et, porta eu, leo. Curabitur vitae sem eget arcu laoreet
  vulputate. Cras orci neque, faucibus et, rhoncus ac, venenatis ac,
  magna. Aenean eu lacus. Aliquam luctus facilisis augue. Nullam
  fringilla consectetuer sapien. Aenean neque augue, bibendum a,
  feugiat id, lobortis vel, nunc. Suspendisse in nibh quis erat
  condimentum pretium. Vestibulum tempor odio et leo. Sed sodales
  vestibulum justo. Cras convallis pellentesque augue. In eu magna. In
  pede turpis, feugiat pulvinar, sodales eget, bibendum consectetuer,
  magna. Pellentesque vitae augue.}
\end{figure*}

\acknowledgements

%\bibliographystyle{apj}

\end{document}
