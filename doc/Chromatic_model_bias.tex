%%%%%%%%%%%%%%%%%%
% DOCUMENT CLASS %
%%%%%%%%%%%%%%%%%%
\documentclass[apj]{emulateapj}

\usepackage{times}
\usepackage{amsmath}
\usepackage{commath}
\usepackage[utf8x]{inputenc}
\urlstyle{same}
\usepackage{hyperref}
\hypersetup{
  pdftitle={Chromatic Model Fitting Bias},
  pdfauthor={Joshua Meyers},
  colorlinks=true,
  linkcolor=blue,
  citecolor=blue,
  urlcolor=blue,
  bookmarksnumbered
}

\pagestyle{myheadings}

\shorttitle{Chromatic Model Fitting Bias}
\shortauthors{Meyers et al.}

\begin{document}

\title{Chromatic Model Fitting Bias}

\author{
J.~Meyers\altaffilmark{1},
The Authors}
\email{jmeyers3@stanford.edu}

\altaffiltext{1}{Department of Physics, Stanford University, Stanford, CA 94305}

%Abstract
\begin{abstract}
This is the abstract.
\end{abstract}

%intro
\section{Introduction}\label{sec:intro}
Cosmic shear experiments aim to constrain cosmological parameters by
measuring the small departure from statistical isotropy of distant
galaxy shapes induced by the gravitational lensing from foreground
large scale structure.  The shapes of galaxy images collected from
telescopes, however, are not only affected by cosmic shear (typically
a $\lesssim$ 1\% effect), but also by the combined point spread
function (PSF) of the atmosphere (for ground-based experiments),
telescope optics, and the image sensor (together often a few \%
effect).  The shape of this additional convolution kernel is typically
constrained from the shapes of stars, which are effectively point
sources before being smeared by the PSF.  Galaxy images can then be
deconvolved with the estimated convolution kernel.  Implicit in this
approach is the assumption that the galactic kernel is the same as the
stellar kernel.  Effects that make the PSF dependent on wavelength
will violate this assumption, as stars and galaxies at different
redshifts have different spectral energy distributions, and hence
different PSFs.

Examples of wavelength-dependent PSF contributions include:

\begin{itemize}
  \item atmospheric differential chromatic refraction (DCR)
  \item atmospheric seeing
  \item telescope optics DCR
  \item sensor photoconversion depth and subsequent charge diffusion
\end{itemize}

This note will focus on PSF mis-estimations due to the first of the
above effects, though the conclusions are applicable to any situation
in which the bias due to using the wrong PSF is estimated using a
``ring test'' with a simple model parameterization.

\section{Analytic Expectations}\label{sec:analytic}
\citet[][hereafter PB12]{Plazas2012} derived analytic expressions for
the bias expected in weak lensing measurements due to DCR.  We
summarize the main points of their argument here.

Let $R(\lambda; z_a)$ be the refraction towards the zenith of a photon
with wavelength $\lambda$ and true zenith angle (before refraction)
$z_A$.  The refraction can be factored into
\begin{equation}
  R(\lambda; z_a) =  h(\lambda) \mathrm{tan}(z_a),
  \label{eqn:Rlamza}
\end{equation}
where $h(\lambda)$ implicitly depends on air pressure, temperature,
and the partial pressure of water vapor, and can be obtained from
\citet{Edlen1953} and \citet{Coleman1960}.  For monochromatic sources,
the only effect is to move the apparent position of the object.  For
sources that are not monochromatic (i.e., all real sources), having a
wavelength density of surviving photons (i.e., the product of the
source photon density and the total system throughput function) given
by $p_{\lambda}(\lambda)$, the mismatched displacements of different
wavelengths introduces a convolution kernel in the zenith direction
that can be written in terms of the inverse of Equation
\ref{eqn:Rlamza}:
\begin{equation}
  k(R) = \frac{p_\lambda(\lambda(R; z_a)) \left|\od{\lambda}{R}\right|}{\int{p_\lambda(\lambda) \dif{\lambda}}}
  \label{eqn:convker}
\end{equation}
This kernel can largely be characterized in terms of its first and
second central moments given by:
\begin{equation}
  \bar{R} = \int{k(R) R \dif{R} } = \frac{\int{R(\lambda; z_a) p_\lambda(\lambda)\dif{\lambda} }}{\int{p_\lambda(\lambda) \dif{\lambda}}}
  \label{eqn:rbar}
\end{equation}
\begin{equation}
  V = \int{k(R) (R - \bar{R})^2 \dif{R} } = \frac{\int{(R(\lambda; z_a) - \bar{R})^2 p_\lambda(\lambda)\dif{\lambda} }}{\int{p_\lambda(\lambda) \dif{\lambda}}}
  \label{eqn:V}
\end{equation}

Galaxy shapes can be characterized by the quadrupole moments of their
light distribution given by:
\begin{equation}
  I_{\mu \nu} = \frac{1}{f}\int{\dif{x} \dif{y} I(x,y)(\mu - \bar{\mu})(\nu - \bar{\nu})}
\end{equation}
\begin{equation}
  \bar{\mu} = \frac{1}{f}\int{\dif{x} \dif{y} I(x,y)\mu}
\end{equation}
\begin{equation}
  f = \int{\dif{x} \dif{y} I(x,y)}
\end{equation}

In particular, galaxy size and 2-component ellipticity are frequently defined as:
\begin{equation}
  r^2 = I_{xx} + I_{yy}
\end{equation}
\begin{equation}
  e_1 = \frac{I_{xx} - I_{yy}}{r^2}
\end{equation}
\begin{equation}
  e_2 = \frac{2 I_{xy}}{r^2}
\end{equation}

For the case of DCR, we can set, without loss of generality, the $x$
direction to be towards zenith.  The effect of DCR is then to take
$I_{xx} \rightarrow I_{xx} + V$, which also takes $r^2 \rightarrow r^2
+ V$, but leaves $I_{yy}$ and $I_{xy}$ unchanged.  The ellipticity and
size parameterizations are defined in terms of the quadrupole moments
before the galaxy light distribution is smeared by the PSF, but we
only have access to the light distribution after convolution.  The
second moments before ($I^g$) and after ($I^o$) convolution are
related via the second moments of the PSF ($I^*$) like $I^g = I^o -
I^*$.  If the SEDs of galaxies and stars were the same, then the
effect of DCR would be to add $V$ to both $I^o$ and $I^*$, which would
then cancel when computing $I^g$ and subsequently galaxy shape
parameters.  The differences between stellar and galactic SEDs,
however, will introduce a small error $\Delta V \ll r^2$ into $I_{xx}$
and $r^2$, which leads to biases in the derived ellipticity
parameters:
\begin{equation}
  e_1 \rightarrow \frac{I_{xx} + I_{yy} + \Delta V}{r^2 + \Delta V} \approx e_1 \left(1 - \frac{\Delta V}{r^2}\right) + \frac{\Delta V}{r^2}
\end{equation}
\begin{equation}
  e_2 \rightarrow \frac{2 I_{xy}}{r^2 + \Delta V} \approx e_2 \left(1 - \frac{\Delta V}{r^2}\right)
\end{equation}

Under the expectation that $\left<e_i\right> \approx 2\gamma_i$ and
defining the shear calibration parameters $m$ and $c$ such that
$\hat{\gamma} = \gamma (1 + m) + c$, where $\hat{\gamma}$ indicates
the estimator for the true shear $\gamma$, we reach the expectation
that $m_1 = m_2 = m = -\frac{\Delta V}{r^2}$, $c_1 = -m/2$, and $c_2 =
0$.  Note that nowhere in the above analysis is there any assumption
on the profile of the galaxy in question.

\section{Ring test}\label{sec:ringtest}
An alternative way to estimate the bias in $\hat{\gamma}$ induced by
DCR is to simulate galaxy images using the ``true'' (galactic) PSF and
then attempt to recover the simulated galactic ellipticities while
using the ``wrong'' (stellar) PSF.  A ring test \citep{Nakajima2007}
is a specific prescription for such a suite of simulations designed to
rapidly converge to the correct (though biased) average value of
$\hat{\gamma}$.  The test gets its name from the arrangement of galaxy
shapes used in the simulated images, which form a ring in ellipticity
space centered at the origin (i.e., $|e|$ is constant), before any
shear is applied.  By choosing intrinsic ellipticities that exactly
average to zero, the results of the test converge faster than for
randomly (but isotropically) chosen ellipticities that only average to
zero statistically.

The general procedure can be implemented as follows:

\begin{enumerate}
  \item Choose an input ``true'' reduced shear $g$
  \item Choose a pre-sheared ellipticity $e^s = (e_1^s, e_2^s)$ on the ring
  \item Compute the sheared ellipticity from $e^o = \frac{e^s+g}{1+g^*e^s}$
  \item Generate a ``truth'' image by convolving the galaxy model
    having ellipticity $e^o$ with the ``true'' PSF (galactic in this
    case)
  \item Using a different ``reconstruction'' PSF for convolution
    (stellar in this case), search the model parameter space to find
    the best fitting model image to the ``truth'' image, and record
    the measured ellipticity from the model parameters
  \item Repeat steps 3-5 using the opposite pre-sheared ellipticity
    $e^s \rightarrow -e^s$
  \item Repeat steps 2-6 for as many values around the ellipticity
    ring as desired.  Typically these are uniformly spaced around the
    ring.
  \item Average all recorded output ellipticity values.  This is the
    shear estimate $\hat{g}$
  \item Repeat steps 1-8 to map out the relation $g \rightarrow
    \hat{g}$
  \item $1+m$ and $c$ are then the slope and intercept of the best-fit
    linear relation between $g$ and $\hat{g}$ (note we assume that $g
    \approx \gamma$)
\end{enumerate}

As described above, the ring test requires some prescription for
creating galaxy images with a given ellipticity.  Here we investigate
using a single Sersic profile as the galaxy model.  The Sersic profile
has 7 parameters: the $x$ and $y$ coordinates of the center, the total
flux, the effective radius $r_e$ (also called the half-light radius),
the two component ellipticity $\mathbf{e}$, and the Sersic index $n$.
Using $r$ as an elliptical radial coordinate, the profile shape is:
\begin{equation}
  I(r) \propto \mathrm{e}^{[-k (r/r_e)^2]^{\frac{1}{2 n}}}
\end{equation}
The constant $k \approx 1.9992 n - 0.3271$ is chosen such that $r_e$
is the half-light radius for a circularized profile.  Specific cases
of the Sersic profile include the Gaussian profile which has $n=0.5$,
an exponential profile which has $n=1.0$, and a de Vaucouleurs profile
which has $n=4.0$.  The profile ranges from smoothly peaked with small
tails at low $n$ (such as an $n=0.5$ Gaussian), to very sharply peaked
with heavy tails (such as an $n=4.0$ de Vaucouleurs profile).

In addition to a galaxy model, the ring test also requires a model for
the PSF.  We have described above the PSF kernel that describes the
contribution of DCR, but other contributions to the PSF also exist,
due to atmospheric turbulence, telescope optics, and the detector for
example.  Ground based telescope PSFs are usually dominated by
atmospheric turbulence, and are frequently modeled by a Moffat
profile:
\begin{equation}
  I_p(r) \propto \left(1+\left(\frac{r}{\alpha}\right)^2\right)^{-\beta}
\end{equation}
\begin{equation}
  \alpha = \frac{\mathrm{FWHM}}{2\sqrt{2^{1/\beta}-1}}
\end{equation}
The PSF used in the ring test in this note is based on a Moffat
profile with a FWHM of $0$\farcs$7$ and softening parameter $\beta =
2.6$, values which were chosen based on fits to simulated LSST stars
via the photon-by-photon simulation package phoSim.  This base PSF is
then convolved (in the zenith direction only) with the DCR kernel
given in Equation \ref{eqn:convker}.  The ``true'' PSF uses a DCR
kernel derived from a galaxy SED, while the ``reconstruction'' PSF
uses a DCR kernel derived from a stellar SED.

\section{Sersic index dependence}\label{sec:sersic}

The results of the ring test for a set of Sersic galaxies simulated at
a zenith angle of 30 degrees are shown in Figure \ref{fig:mandc}.  The
galaxy size quantified by $r^2$ is the same for each Sersic index.
The ring test results match the analytic formulae well when Gaussian
galaxies ($n=0.5$) are used, but diverge by factors of 2-3 when de
Vaucouleurs profiles ($n=4.0$) are used.  The PB12 analytic formalism
has no dependence on the profile of the galaxy, but only on the
assumption that the difference between stellar and galactic dispersion
kernels is much smaller than the size of the unsmeared galaxy ($\Delta
V \ll r^2$).  This condition is met identically for each profile by
construction, suggesting either that something is flawed with the PB12
formalism or with the ring test.

\begin{figure*}
\begin{center}
\epsscale{1.0}
\plotone{figures/all_pb12_ring_n.pdf}
\end{center}
\caption[fig1]{\label{fig:mandc} The shear calibration parameters $m$
  and $c$ plotted against redshift.  The results are obtained from the
  ring test assuming a true PSF derived from an elliptical galaxy
  spectrum and reconstruction PSF derived from a G5v stellar spectrum.
  The true galaxy images are simulated using the values: $r_e = 1.1$,
  and $|e| = 0.2$.  Three values of the galaxy Sersic parameter $n$
  are shown: $n=0.5$, corresponding to a Gaussian profile (blue),
  $n=1.0$, corresponding to an exponential profile (green), and
  $n=4.0$, corresponding to a de Vaucouleurs profile (red).  Squares
  are used for $m_1$ and $c_1$, and ``x's'' for $m_2$ and $c_2$. The
  cyan line in each panel shows the analytic result from the PB12
  formalism.  The zenith angle is 30 degrees.}
\end{figure*}

\section{Ring test case study}\label{sec:casestudy}

To investigate the ring test in more detail, we have conducted a case
study focusing on the redshift 1.25 galaxy that shows the largest $m$
and $c$ values in both the ring test results and the PB12 results.
For this study, we have constructed a series of diagnostic images
showing the results of the fits in Step 5 of the ring test for each of
the Sersic indices studied.  These diagnostic images are shown in
Figures \ref{fig:gauss}, \ref{fig:exp}, and \ref{fig:deV}.  Note that
the diagnostic figures were generated for a zenith angle of 60 degrees
for better visualization, though the qualitative results are the same
at 30 degrees.

In Figures \ref{fig:gauss}, \ref{fig:exp}, and \ref{fig:deV}, the
``truth'' image is shown in the top right panel.  The panel
immediately below this one is the best fit to the ``truth'' image
possible using the ``reconstruction'' PSF, which in this case is the
PSF generated from the G5v stellar spectrum.  The leftmost panel in
the second row shows the above-atmosphere galaxy model used to create
this ``best fit'' image.  The rightmost panel in the third row shows
the residual between the ``truth'' image and the ``best fit'' image,
and is an indication of the quality of the fit in Step 5 of the ring
test.  If the fit were perfect, then the residual image would be
entirely zeros.

The reason the residual image is not entirely filled with zeros is due
to ``model bias'' \citep{Melchior2009, Voigt2010, Bernstein2010}.  The
fitting step can be viewed as an attempt at deconvolution of the
``truth'' image by the ``reconstruction'' PSF under the assumption
that the functional form of the deconvolved image is known.  The best
fit model (second row, first column) is the result of this attempted
deconvolution, and the ellipticity parameters can then simply be read
off as the parameters that generated the model.  In practice, however,
the true deconvolved image (i.e. the true deconvolution of the
``truth'' image by the ``reconstruction'' PSF, not computed or shown
in any of the Figure panels) will not always be an instance of the
functional form of the fit.  More generally, the true deconvolution
may not even have elliptical isophotes, precluding a solution by
simply adding degrees of freedom to the radial profile of the
functional form.  (Degrees of freedom can, of course, be added {\it
  azimuthally} to get a perfect deconvolution, but then the
ellipticity can no longer be simply read off as a model parameter).

The degree to which the ring test results are biased by model fitting
depends on how well the deconvolution is able to function.  Larger
residuals to the fit are more susceptible to bias.  Comparing Figures
\ref{fig:gauss}, \ref{fig:exp}, and \ref{fig:deV}, we see visually
that the residuals grow with the Sersic index, which explains how the
$n=4.0$ curve of Figure \ref{fig:mandc} became so different than the
analytic prediction.  All of the curves are subject to both model bias
and bias from DCR, but the $n=4.0$ curve is the most sensitive to
model bias.  The visual comparison is backed up by the data in Table
\ref{table:fitqual}.  As the Sersic index increases, the best fit
$\chi^2$ also increases indicating a worse fit.  From the quadrupole
moments of the residuals, one can define a Figure of Demerit:
\begin{equation}
  \mathrm{FoD} = I^r_{xx} + I^r_{yy}
\end{equation}
where $I^r$ is the residual image.  The Figure of Demerit also
increases with Sersic index, indicating an increasing model bias.

\begin{deluxetable*}{lllllll}
\tablecaption{\label{table:fitqual} Fit Quality}
\tablehead{
\colhead{n} &
\colhead{$\chi^2$} &
\colhead{FoD} &
\colhead{m} &
\colhead{c} &
\colhead{PB12 m} &
\colhead{PB12 c}}
\startdata
0.5 & 2.08e-4 & 5.82e-10 & (0.008672, 0.008513) & (0.004213, 0.0) & -0.008111 & 0.004055 \\
1.0 & 3.28e-3 & 8.97e-9  & (0.010016, 0.009828) & (0.005755, 0.0) & -0.008111 & 0.004055 \\
4.0 & 1.15e-2 & 9.01e-8  & (0.023039, 0.022608) & (0.020024, 0.0) & -0.008111 & 0.004055
\enddata
\end{deluxetable*}


Presumably, better ellipticity measurement algorithms would show less
dependence on the Sersic index.  For instance, the algorithms
described in \citet{Bernstein2010} and \citet{Melchior2011} both aim
to mitigate model bias.  If Step 5 of the ring test were replaced with
one of these algorithms for measuring ellipticity, then maybe the
effect of DCR could better be isolated from the effects of model bias.

\begin{figure*}
\begin{center}
\epsscale{1.0}
\plotone{figures/case_study_n05.pdf}
\end{center}
\caption[fig2]{\label{fig:gauss} Illustration of the simulation and
  fitting portion of the ring test procedure (steps 4 and 5).  The
  columns from left to right represent (i) the high-resolution model
  galaxy before PSF convolution, (ii) the PSF, (iii) the galaxy
  convolved with the PSF, and (iv) the pixelization of (iii).  The
  rows from top to bottom represent (i) the ``true'' galaxy model,
  PSF, and convolutions, (ii) the best fit galaxy model, PSF, and
  convolutions, (iii) the base-10 logarithm of the absolute value of
  the residual ($\mathrm{best fit} - \mathrm{truth}$), and (iv) the
  base-10 logarithm of the absolute fractional residual
  ($(\mathrm{best fit} - \mathrm{truth})/\mathrm{truth}$).  The fit is
  designed to minimize the pixelized residuals in the third row,
  fourth column.  The difference between the truth and best fit galaxy
  models illustrates the size of the model bias induced by using the
  ``reconstruction'' PSF in the fit.  The galaxy profile is Gaussian
  in this figure ($n = 0.5$, both in truth image and fixed during the
  fit), the galaxy PSF is derived from an elliptical spectrum at
  redshift 1.3, and the stellar PSF is derived from the spectrum of a
  G5v star.  The zenith angle is 60 degrees.}
\end{figure*}

\begin{figure*}
\begin{center}
\epsscale{1.0}
\plotone{figures/case_study_n10.pdf}
\end{center}
\caption[fig3]{\label{fig:exp} The same as Figure \ref{fig:gauss}, but
  the galaxy profile is changed to be exponential ($n=1.0$, in both
  the truth image and fixed during the fit).}
\end{figure*}

\begin{figure*}
\begin{center}
\epsscale{1.0}
\plotone{figures/case_study_n40.pdf}
\end{center}
\caption[fig4]{\label{fig:deV} The same as Figure \ref{fig:gauss}, but
  the galaxy profile is changed to be de Vaucouleurs ($n=4.0$, in both
  the truth image and fixed during the fit). }
\end{figure*}

\acknowledgements

\bibliographystyle{apj}
\bibliography{Chromatic_model_bias}

\end{document}
